\section{様々な帯電体による電界,電位(p.78)}
\subsection{一様に帯電した球(p.82)}
\subsubsection{電界}
\pic{./files/1/imgs/12.pdf}{0.2}
\ref{ChargedBall}と同様に,半径$a[m]$の球に$Q[C]$の電荷が一様に分布している球と同じ中心をもつ半径$r[m]$の球面上の電界の大きさを$E$とする.\\
{\bf ・外部$(a < r)$}\\
ガウスの法則より
\begin{flalign}
&\oint_{S_{0}}EdS=\dfrac {Q}{\varepsilon _{0}}\\
&E\cdot 4\pi r^{2}=\dfrac {Q}{\varepsilon _{0}}\\
&\therefore E=\dfrac {Q}{4\pi \varepsilon _{0}r^{2}}\left[ V/m\right]
\end{flalign}

{\bf ・内部$(a < r)$}\\
\begin{table}[htb]
  \begin{center}
    \begin{tabular}{|l|c|c|} \hline
      & 体積 & 電荷 \\ \hline
      全体 & $\dfrac {4}{3}\pi a^{3}$ & $Q$\\
      求めるところ & $\dfrac {4}{3}\pi r^{3}$ & $Q\dfrac {r^{3}}{a^{3}}$\\ \hline
    \end{tabular}
  \end{center}
\end{table}
\begin{eqnarray}
  \because \dfrac {Q}{\dfrac {4}{3}\pi a^{3}}\cdot \dfrac {4}{3}\pi r^{3}=Q\cdot \dfrac {r^{3}}{a^{3}}
\end{eqnarray}

ガウスの法則より
\begin{flalign}
&\oint _{S_{0}}EdS=\dfrac {Q\dfrac {r^{3}}{a^{3}}}{\varepsilon _{0}}\\
&E\cdot 4\pi r^{2}=\dfrac {Qr^{3}}{\varepsilon _{0}a^{3}}\\
&E=\dfrac {Q}{4\pi \varepsilon _{0}a^{3}}r
\end{flalign}
\pic{./files/1/imgs/13.pdf}{0.2}

\subsubsection{電位}
誘電体球の中心から$R[m]$離れた点の電位を$V_R$とする.\\
{\bf ・外部$(a < r)$}\\
\begin{eqnarray}
V_{R}&=&-\int ^{R}_{\infty }Edr\\
&=&-\int ^{R}_{\infty }\dfrac {Q}{4\pi \varepsilon _{0}r^{2}}dr\\
&=&\dfrac {Q}{4\pi \varepsilon _{0}}\int ^{R}_{\infty }\left( -r^{2}\right) dr\\
&=&\dfrac {Q}{4\pi \varepsilon _{0}}\left[ r^{-1}\right] ^{R}_{\infty }\\
&=&\dfrac {Q}{4\pi \varepsilon _{0}}\left( \dfrac {1}{R}-0\right) \\
&=&\dfrac {Q}{4\pi \varepsilon _{0}R}\left[ V\right]
\end{eqnarray}

{\bf ・内部$(a < r)$}\\
\begin{eqnarray}
  V_{R}&=&-\int ^{R}_{\infty }Edr\\
  &=&-\left( \int ^{a}_{\infty }Edr+\int ^{R}_{a}Edr\right) \\
  &=&\dfrac {Q}{4\pi \varepsilon _{0}a}-\dfrac {Q}{4\pi \varepsilon _{0}a^{3}}\times \int ^{R}_{a}rdr\\
  &=&\dfrac {Q}{4\pi \varepsilon _{0}a}-\dfrac {Q}{4\pi \varepsilon _{0}a^{3}}\times \left[ \dfrac {1}{2}r^{2}\right] ^{R}_{a}dr\\
  &=&\dfrac {Q}{4\pi \varepsilon _{0}a}-\dfrac {Q}{4\pi \varepsilon _{0}a^{3}}\times \dfrac {1}{2}\left( R^{2}-a^{2}\right)\\
  &=&\dfrac {Q}{4\pi \varepsilon _{0}a}-\dfrac {Q}{8\pi \varepsilon _{0}a^{3}}R^{2}+\dfrac {Q}{8\pi \varepsilon _{0}a^{3}}\\
  &=&\dfrac {Q}{4\pi \varepsilon _{0}a}-\dfrac {Q}{8\pi \varepsilon _{0}a^{3}}R^{2}+\dfrac {Q}{8\pi \varepsilon _{0}a^{3}}\\
  &=&\dfrac {Q}{4\pi \varepsilon _{0}a}\left( 1-\dfrac {R^{2}}{2a^{2}}+\dfrac {1}{2}\right)\\
  &=&\dfrac {Q}{4\pi \varepsilon _{0}a}\left( \dfrac {3}{2}-\dfrac {R^{2}}{2a^{2}}\right) \left[ V\right]
\end{eqnarray}

Rについて,\\
内部の電位は上に凸の放物線($R=0$の時に最大値をとる)\\
外部の電位は分数関数\\
したがって,グラフは
\pic{./files/3/imgs/1.pdf}{0.2}

\subsection{表面が一様に帯電した導体球(p.87)}
\subsubsection{電界}
\pic{./files/1/imgs/14.pdf}{0.2}
半径$a[m]$の導体球の表面上に$Q[C]$の電荷が一様に分布している球\\
導体球と同じ中心をもつ半径$r[m]$の球面上の電界の大きさを$E$とする.\\

{\bf ・外部$( r > a)$}\\
ガウスの法則より
\begin{flalign}
&\oint_{S_{0}}EdS=\dfrac {Q}{\varepsilon _{0}}\\
&E\cdot 4\pi r^{2}=\dfrac {Q}{\varepsilon _{0}}\\
&\therefore E=\dfrac {Q}{4\pi \varepsilon _{0}r^{2}}\left[ V/m\right]
\end{flalign}

{\bf ・表面$(r = a)$}\\
\begin{eqnarray}
E&=&\dfrac {\sigma }{\varepsilon _{0}}=\dfrac {\dfrac {Q}{4\pi a^{2}}}{\varepsilon _{0}}\\
 &=&\dfrac {Q}{4\pi \varepsilon _{0}a^{2}}\left[ V/m\right]
\end{eqnarray}

{\bf ・内部$(0\leqq r <a)$}\\
\begin{flalign}
&\oint _{s_{0}}EdS=\dfrac {0}{\varepsilon _{0}}\\
&E \times 4\pi r^{2}=\dfrac {0}{\varepsilon _{0}}=0\\
&\therefore E=0\left[ V/m\right]
\end{flalign}

外部と表面について,違う方法で求めたが,結果が同じになったので,同じ関数で表せるものとして扱う.\\
電界のグラフは,
\pic{./files/1/imgs/15.pdf}{0.2}

\subsubsection{電位}
{\bf ・表面,外部$(a \leqq r)$}\\
\begin{eqnarray}
V_{R}&=&-\int ^{R}_{\infty }Edr\\
&=&-\int ^{R}_{\infty }\dfrac {Q}{4\pi \varepsilon _{0}r^{2}}dr\\
&=&\dfrac {Q}{4\pi \varepsilon _{0}R}\left[ V\right]
\end{eqnarray}

{\bf ・内部$(0\leqq r <a)$}\\
\begin{eqnarray}
  V_{R}&=&-\int ^{R}_{\infty }Edr\\
  &=&-\left( \int ^{a}_{\infty }Edr+\int ^{R}_{a}Edr\right) \\
  &=&\dfrac {Q}{4\pi \varepsilon _{0}a}- \int ^{R}_{a}0dr\\
  &=&\dfrac {Q}{4\pi \varepsilon _{0}a}\left[ V\right]
\end{eqnarray}

電位のグラフは,
\pic{./files/3/imgs/2.pdf}{0.2}

{\bf $R$が$0$から離れていく時,電位が大きくなることはない}\\

このような問題を解くときのポイント\\
1.電荷の分布を決める\\
2.電界を内側から決める\\
3.電位は必ず無限遠点から決める\\

\subsection{電気双極子(p.78)}
\subsubsection{電気双極子}
\pic{./files/3/imgs/3.pdf}{0.25}
電気双極子:大きさが等しく,符号が逆の極めて接近して存在するもの.

\subsubsection{電気双極子により作られる電位}
\pic{./files/3/imgs/4.pdf}{0.3}
電位は各点電荷による電位の和をとれば良い\\
$+Q$,$-Q$の点電荷による電位$V_p$は
\begin{eqnarray}
  V_p &=& \dfrac{Q}{4\pi \varepsilon _{0}\overline{AP}} + \dfrac{-Q}{4\pi \varepsilon _{0}\overline{BP}}\\
  &=& \dfrac{Q}{4\pi \varepsilon _{0}\overline{AP}} - \dfrac{Q}{4\pi \varepsilon _{0}\overline{BP}}\\
  &=& \dfrac{Q}{4\pi \varepsilon _{0}} \cdot \dfrac{\overline{BP}-\overline{AP}}{\overline{AP}\cdot \overline{BP}}\\
  &\fallingdotseq& \dfrac{Q}{4\pi \varepsilon _{0}} \cdot \dfrac{l \cdot \cos(\theta)}{r^2}\\
  &=& \dfrac{Ql \cos(\theta)}{4\pi \varepsilon _{0}r^2}\left[ V\right]
\end{eqnarray}

\subsubsection{双極子モーメント}
ここで,$P=Ql[Cm]$と置くと,電気双極子により作られる電位は次のようになる.
\begin{flalign}
  &V_p = \dfrac{P}{4\pi \varepsilon _{0}r^2} \cos(\theta)\left[ V\right]
\end{flalign}

この$P$を双極子モーメントと呼ばれ,次のように定義される.
大きさ:$P=Ql[Cm]$\\
向き:負から正の向き\\
