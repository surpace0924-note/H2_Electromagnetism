\section{電位}
\subsection{電界中で電荷を移動
するのに要する仕事}
\pic{./files/2/imgs/1.pdf}{0.25}
力$F[N]$が物体にした仕事$W[J]$は
\begin{flalign}
&W = Fl[J]
\end{flalign}\\
ここで移動距離を細かく分けてみる
\begin{eqnarray}
W &=&F(\Delta l+\Delta l+\Delta l+\ldots+\Delta l)\\
&=&F\Delta l+F\Delta l+F\Delta l+\ldots+F\Delta l
\end{eqnarray}

\subsection{電荷を運ぶのに要する仕事}
\subsubsection{ケース1}
\pic{./files/2/imgs/2.pdf}{0.25}
$Q_1$の作る電界が$Q_2$に対して仕事をした.\\
 →仕事の符号は「負」となる.\\

\subsubsection{ケース2}
\pic{./files/2/imgs/3.pdf}{0.25}
外部から$Q_2$に対して仕事をした.\\
 →仕事の符号は「正」となる.\\

\subsection{仕事の計算}
\subsubsection{ケース1}
\pic{./files/2/imgs/4.pdf}{0.23}
\begin{flalign}
&F=\dfrac {1}{4\pi \varepsilon _{0}}\times \dfrac {Q_{1}Q_{2}}{r^{2}}
\end{flalign}\\
$Q_2$の電荷が$r = r_a$から$r=r_b$まで動くとき,$Q_1$による電界が$Q_2$にした仕事$W$の近似値は
\begin{eqnarray}
W&=&-F_{1}\times \Delta r_{1}-F_{2}\times \Delta r_{2}\ldots \\
&=&-\sum ^{3}_{k=1}F_{k}\Delta r_{k}
\end{eqnarray}\\
分割を限りなく細かくすると,積分を用いて$W$を正確に求めることができる.
\begin{flalign}
&W=-\int ^{r_{b}}_{r_{a}}Fdr\left[ J\right]
\end{flalign}

\subsubsection{ケース2}
\pic{./files/2/imgs/5.pdf}{0.25}
クーロン力に逆らって,外部から$Q_2$にした仕事$W$は
\begin{eqnarray}
W&=&\int ^{r_{a}}_{r_{b}}Fdr\left[ J\right] \\
&=&-\int ^{r_{b}}_{r_{a}}Fdr\left[ J\right]
\end{eqnarray}\\

\subsubsection{まとめ}
2つのケースのいずれの場合でも,以下のように表せる.
\begin{flalign}
&W=-\int ^{\mbox{終点}}_{\mbox{始点}}(\mbox{クーロン力})dr\left[ J\right]\\
&\begin{cases}W <0:\mbox{電界がした力}\\
W >0:\mbox{外部がした力}\end{cases}
\end{flalign}

\subsection{電位}
\subsubsection{無限遠点から電荷を運ぶのに要する仕事}
\pic{./files/2/imgs/6.pdf}{0.25}
\begin{eqnarray}
W&=&-\int ^{r_{a}}_{\infty }\dfrac {Q_{1}Q_{2}}{4\pi \varepsilon _{0}r^{2}}dr\\
&=&\dfrac {Q_{1}Q_{2}}{4\pi \varepsilon _{0}}\left[ r^{-1}\right] ^{r_{a}}_{\infty }\\
&=&\dfrac {Q_{1}Q_{2}}{4\pi \varepsilon _{0}}\left( \dfrac {1}{r_{a}}-0\right) \\
&=&\dfrac {Q_{1}Q_{2}}{4\pi \varepsilon _{0}r_{a}}\left[ J\right]
\end{eqnarray}\\

\subsubsection{無限遠点から$+1C$の電荷を運ぶのに要する仕事}
前問において,$Q_2 = 1[C]$とすればよい.
\begin{eqnarray}
\therefore W&=&\dfrac {Q_{1}}{4\pi \varepsilon _{0}r_{a}}\left[ J\right] \\
&=&-\int ^{r_{a}}_{\infty }\dfrac {Q_{1}Q_{2}}{4\pi \varepsilon _{0}r^{2}}dr\\
&=&-\int ^{r_{a}}_{\infty }EQ_{2}dr
\end{eqnarray}\\

ここで$Q_2 = 1[C]$だから
\begin{flalign}
&W=-\int ^{r_{a}}_{\infty }Edr
\end{flalign}

\subsubsection{電位}
\begin{itembox}[l]{電位}
無限遠点から$+1C$の電荷を運ぶのに要する仕事を$r=r_a$における電位と定義する.\\
単位は$[V]$を使用する.
\begin{flalign}
&V_a=-\int ^{r_{a}}_{\infty }Edr\left[ V\right]
\end{flalign}
\end{itembox}\\

\subsubsection{電位の基準}
1. 電気磁気学:無限遠点\\
2. 電力   :大地\\
3. 回路   :任意

\subsubsection{点電荷による電界}
\pic{./files/2/imgs/7.pdf}{0.1}
点$P$の電位は
\begin{flalign}
&V_{a}=\dfrac {Q}{4\pi \varepsilon _{0}r_{a}}\left[ V\right]
\end{flalign}

\subsection{電位差}
\subsubsection{電位差}
\pic{./files/2/imgs/8.pdf}{0.4}
$V_a-V_b$を点$B$に対する点$A$の電位差という.\\
 →点$B$から点$A$に$+1C$を運ぶのに要する仕事.\\
電位差を電圧ともいう.

\subsubsection{計算方法}
点$B$に対する点$A$の電位差$V_{AB}$は\\
\begin{flalign}
&V_{AB}=-\int ^{r_{a}}_{r_b}(\mbox{電界})dr
\end{flalign}\\

原点に$Q[C]$がある場合は\\
\begin{eqnarray}
V_{AB}&=&-\int ^{r_{a}}_{r_{b}}\dfrac {Q}{4\pi \varepsilon _{0}}dr\\
&=&\dfrac {Q}{4\pi \varepsilon _{0}}\left[ r^{-1}\right] ^{r_{a}}_{r_{b}}\\
&=&\dfrac {Q}{4\pi \varepsilon _{0}}\left( \dfrac {1}{r_{a}}-\dfrac {1}{r_{b}}\right) \left[ V\right]
\end{eqnarray}

\subsubsection{等電位線と電気力線の関係}
\pic{./files/2/imgs/9.pdf}{0.4}
・電界中で,電位$V$の等しい点を連ねて作った仮想的な線を等電位線という.\\
・二次元の時は等電位線,三次元の時は等電位面.\\
・等電位線と電気力線は垂直に交わる.\\
・電位の山を下る時,最も急な傾斜が電界の向き.\\
・等電位線に沿って電荷を動かした時,仕事はしない.\\
・動かす方向に力はない.

\subsection{電位の傾き}
\subsubsection{勾配}
\pic{./files/2/imgs/10.pdf}{0.15}
\begin{flalign}
&\mbox{平均変化率}=\dfrac {\Delta y}{\Delta x}\\
&\mbox{勾配}=\dfrac {dy}{dx}\left[ (\mbox{yの単位})/m\right]
\end{flalign}


\subsubsection{電位の傾き}
\pic{./files/2/imgs/11.pdf}{0.25}
$Q$による電界$E$が行った仕事$\Delta W$は
\begin{eqnarray}
\Delta W&=&-(\mbox{力})\times (\mbox{距離})\\
&=&-qE\times \Delta x
\end{eqnarray}\\

この時,$E$が一定とみなせるほど$\Delta x$は小さい.\\
$q=1C$なら,$\Delta W$は電位差$\Delta V$と置き換えられる.
\begin{flalign}
&\Delta V=-E \times \Delta x\\
&\dfrac {\Delta V}{\Delta x}=-E\\
&E=-\dfrac {\Delta V}{\Delta x}\left[ V/m\right]
\end{flalign}

$\Delta x$が限りなく小さければ
\begin{flalign}
&E=-\dfrac {dV}{dx}\left[ V/m\right]\left[ N/C\right]
\end{flalign}\\

すなわち,電界は電位の勾配にマイナスをつけたもの.\\
 →$[N/C]=[V/m]$
