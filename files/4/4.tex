\section{静電容量(p.99)}
\subsection{静電容量(p.102)}
\subsubsection{ケース1}
\pic{./files/4/imgs/dummy.png}{0.2}
導体に$Q[C]$の電荷を与えたとき,電圧$V[V]$
\begin{flalign}
  &C=\dfrac{Q}{V} \left[ F\right]
\end{flalign}
このときの$C$を静電容量という.

\subsubsection{ケース2}
\pic{./files/4/imgs/dummy.png}{0.2}
導体A,Bにそれぞれ$+Q[C]$,$-Q[C]$の電荷を与えたとき,Bに対するAの電位差を$V_{AB}$とすると
\begin{flalign}
  &C=\dfrac{Q}{V_{AB}} \left[ F\right]
\end{flalign}
これをAB間の静電容量という.

\subsection{静電容量の計算(p.104)}
\subsubsection{導体球の静電容量(p.104)}
\pic{./files/4/imgs/dummy.png}{0.2}
{\bf ・電界}\\
{\bf ・外部$(a < r)$}\\
\begin{flalign}
  &E=\dfrac {Q}{4\pi \varepsilon _{0}r^{2}}\left[ V/m\right]
\end{flalign}
{\bf ・表面$(a = r)$}\\
\begin{flalign}
&E=\dfrac {Q}{4\pi \varepsilon _{0}a^{2}}\left[ V/m\right]
\end{flalign}

{\bf ・電位}\\
\begin{eqnarray}
V&=&-\int ^{a}_{\infty }Edr\\
&=&-\int ^{a}_{\infty }\dfrac {Q}{4\pi \varepsilon _{0}r^{2}}dr\\
&=&\dfrac {Q}{4\pi \varepsilon _{0}}\left[ \dfrac {1}{r}\right] ^{a}_{\infty }\\
&=&\dfrac {Q}{4\pi \varepsilon _{0}a}\left[ V\right]
\end{eqnarray}

{\bf ・静電容量}\\
\begin{eqnarray}
C&=&\dfrac {Q}{V}\\
&=&4\pi \varepsilon _{0}a\left[ F\right]
\end{eqnarray}

\subsubsection{同心球間の静電容量(p.104)}
\pic{./files/4/imgs/dummy.png}{0.2}
{\bf ・電界}\\
\begin{flalign}
&E\times 4\pi r^{2}=\dfrac {Q}{\varepsilon _{0}}\\
&E=\dfrac {Q}{4\pi \varepsilon _{0}r^{2}}\left[ V/m\right]
\end{flalign}

{\bf ・電位差}\\
\begin{eqnarray}
V_{AB}&=&-\int ^{a}_{b}Edr\\
&=&-\int ^{a}_{b}\dfrac {Q}{4\pi \varepsilon _{0}r^{2}}dn\\
&=&\dfrac {1}{4\pi \varepsilon _{0}}\left[ \dfrac {1}{r}\right] ^{a}_{b}\\
&=&\dfrac {1}{4\pi \varepsilon _{0}}\left( \dfrac {1}{a}-\dfrac {1}{b}\right)\\
&=&\dfrac {Q\left( b-a\right) }{4\pi \varepsilon _{0}ab}\left[ V\right]
\end{eqnarray}

{\bf ・静電容量}\\
\begin{eqnarray}
C&=&\dfrac {Q}{V_{ab}}\\
&=&\dfrac {4\pi \varepsilon _{0}ab}{b-a}\left[ F\right]
\end{eqnarray}

\subsubsection{同心円筒間の静電容量(p.106)}
\pic{./files/4/imgs/dummy.png}{0.2}
無限長の同軸ケーブルのイメージ\\
円柱A:$+\lambda [C/m]$\\
円柱B:$-\lambda [C/m]$\\

{\bf ・電界}\\
\begin{flalign}
&E\times 2\pi rl+0\times \pi r^{2}\times 2=\dfrac {\lambda l}{\varepsilon _{0}}\\
&E=\dfrac {\lambda }{2\pi \varepsilon _{0}r}\left[ V/m\right]
\end{flalign}

{\bf ・電位差}\\
\begin{eqnarray}
V_{AB}&=&-\int ^{a}_{b}Edn\\
&=&-\int ^{a}_{b}\dfrac {1}{2\pi \varepsilon _{0}r}dr\\
&=&-\dfrac {\lambda }{2\pi \varepsilon _{0}}\int ^{a}_{b}\dfrac {1}{r}dn\\
&=&-\dfrac {\lambda }{2\pi \varepsilon _{0}}\left[ \log \left| r\right| \right] ^{a}_{b}\\
&=&-\dfrac {\lambda }{2\pi \varepsilon _{0}}\left( \log a-\log b\right) \\
&=&\dfrac {\lambda }{2\pi \varepsilon _{0}}\left( \log b-\log a\right) \\
&=&\dfrac {\lambda }{2\pi \varepsilon _{0}}\log \dfrac {b}{a}\left[ V\right]
\end{eqnarray}

{\bf ・静電容量}\\
単位長さあたりの静電容量を$C\left[ F/m\right]$と置いて
\begin{flalign}
&Q=CV\\
&\lambda l=Cl\cdot V_{AB}\\
&C=\dfrac {\lambda }{V_{AB}}\\
&C=\dfrac {2\pi \varepsilon _{0}}{\log \dfrac {b}{a}}\left[ F/m\right]
\end{flalign}

\subsubsection{並行平板間の静電容量(p.107)}
\pic{./files/4/imgs/dummy.png}{0.2}
導体A:$+\lambda [C]$\\
導体B:$-\lambda [C]$\\
{\bf ・電界}\\
\begin{eqnarray}
E&=&\dfrac {\left| \sigma \right| }{\varepsilon _{0}}\\
&=&\dfrac {\left| \dfrac {Q}{S}\right| }{\varepsilon _{0}}\\
&=&\dfrac {Q}{\varepsilon _{0}S}\left[ V/m\right]
\end{eqnarray}

{\bf ・電位差}\\
\begin{eqnarray}
V_{AB}&=&E\cdot d\\
&=&\dfrac {Qd}{\varepsilon _{0}S}\left[ V\right]
\end{eqnarray}

{\bf ・静電容量}\\
\begin{eqnarray}
C&=&\dfrac {Q}{V_{AB}}\\
&=&\dfrac {Q}{\dfrac {Qd}{\varepsilon _{0}S}}\\
&=&\dfrac {\varepsilon _{0}S}{d}
\end{eqnarray}

\subsubsection{無限長の並行導線間の静電容量(p.108)}
\pic{./files/4/imgs/dummy.png}{0.2}
中心間隔:$d[m]$\\
半径:$a[m]$\\

{\bf ・電界}\\
$x$軸状の電界の大きさ$E$は
\begin{flalign}
&E=\dfrac {\lambda }{2\pi \varepsilon _{0}x}+\dfrac {\lambda }{2\pi \varepsilon _{0}\left( d-x\right) }
\end{flalign}

{\bf ・電位差}\\
\begin{eqnarray}
V_{AB}&=&-\int ^{a}_{d-a}Edx\\
&=&-\dfrac {\lambda }{2\pi \varepsilon _{0}}\int ^{a}_{d-a}\left( \dfrac {1}{x}+\dfrac {1}{dx}\right) dx\\
&=&-\dfrac {\lambda }{2\pi \varepsilon _{0}}\left[ \log \left| x\right| -\log \left| d-x\right| \right] ^{a}_{d-a}\\
&=&-\dfrac {\lambda }{2\pi \varepsilon _{0}} \left( \log a-\log \left( d-a\right) \log \left( d-a\right) +\log a\right) \\
&=&\dfrac {\lambda }{2\pi \varepsilon _{0}}\times 2\left( \log \left( d-a\right) -\log a\right)\\
&=&\dfrac {\lambda }{\pi \varepsilon _{0}}\log \dfrac {d-a}{a}\left[ V\right]
\end{eqnarray}

{\bf ・静電容量}\\
単位長さあたりの静電容量を$C\left[ F/m\right]$と置いて
\begin{flalign}
&\lambda l=ClV_{AB}\\
&C=\dfrac {\lambda }{V_{AB}}\\
&C=\dfrac {\pi \varepsilon _{0}}{\log \dfrac {d-a}{a}}\left[ F/m\right]
\end{flalign}

\subsection{電気影像法(p.117)}
\pic{./files/4/imgs/dummy.png}{0.2}
静電誘導による電界,電荷分布を求める方法.

\pic{./files/4/imgs/dummy.png}{0.2}
\begin{flalign}
&E_{1}=E_{2}=\dfrac {Q}{4\pi \varepsilon _{0}r^{2}}
\end{flalign}

求める電界の大きさを$E$とおく.
\begin{eqnarray}
E&=&E_{1}\sin \theta +E_{2}\sin \theta \\
&=&2E,\sin \theta \\
&=&2\times \dfrac {Q}{4\pi \varepsilon _{0}r^{2}}\cdot \dfrac {a}{r}\\
&=&\dfrac {Qa}{2\pi \varepsilon \cdot r^{3}}\\
&=&\dfrac {Qa}{{\pi \varepsilon _{0}\left( x^{2}+a^{2}\right)}^{ \frac {3}{2}}}\left[ V/m\right]
\end{eqnarray}

ところで,導体表面の電界と電荷密度の関係(クーロンの定理)より
\begin{flalign}
&E=\dfrac {\sigma }{\varepsilon _{0}}\left[ V/m\right]\\
&\sigma =E\varepsilon _{0}
\end{flalign}

よって
\begin{eqnarray}
\sigma &=& E\varepsilon _{0}\\
&=&\dfrac {Qa}{2\pi \left( x^{2}+a^{2}\right) ^{\frac {3}{2}}}\left[ C/m^{2}\right]
\end{eqnarray}

\subsection{コンデンサに蓄えられるエネルギー(p.125)}
電位が$3V$
 →無限遠点から$+1C$の電荷を運ぶのに必要な仕事が$3[J]$

\pic{./files/4/imgs/dummy.png}{0.2}
電位が$frac{q}{C}$の導体に$\Delta q[C]$の電荷を運ぶのに必要な仕事を$\Delta W[J]$と置くと
\begin{flalign}
&\Delta W=\dfrac {q}{C}\Delta q
\end{flalign}

電荷量が$0[C]$から$Q[C]$になるまでの仕事$W$は
\begin{eqnarray}
W&=&\sum \Delta W\\
&=&\sum \dfrac {q}{C}\Delta q\\
&=&\int ^{Q}_{0}\dfrac {q}{C}dq\\
&=&\left[ \dfrac {q^{2}}{2C}\right] ^{Q}_{0}\\
&=&\dfrac {Q^{2}}{2C}\left[ J\right] \\
&=&\dfrac {Q\cdot CV}{2C}\\
&=&\dfrac {Q\cdot V}{2}\\
&=&\dfrac {1}{2}CV^{2}\left[ J\right] \\
\end{eqnarray}

導体に蓄えられるエネルギー
 →2つの導体からなるコンデンサに蓄えられるエネルギーもこの形になる.

\subsection{電界に蓄えられるエネルギーの密度(p.127)}
\pic{./files/4/imgs/dummy.png}{0.2}
並行平板コンデンサに蓄えられるエネルギー$W[J]$は
\begin{eqnarray}
W&=&\dfrac {1}{2}CV^{2}\\
&=&\dfrac {1}{2}\dfrac {\varepsilon _{0}S}{d}\times \left( Ed\right) ^{2}\\
&=&\dfrac {1}{2}\varepsilon SdE^{2}\left[ J\right]
\end{eqnarray}

電極の体積で割ったものを$w\left[ J/m^{3}\right]$と置くと
\begin{eqnarray}
w&=&\dfrac {W}{Sd}\\
&=&\dfrac {\dfrac {1}{2}\varepsilon _{0}SdE^{2}}{Sd}\\
&=&\dfrac {1}{2}\varepsilon _{0}E^{2}\left[ J/m^{3}\right]
\end{eqnarray}
これは,電界に蓄えられている電気的エネルギーの密度を表している.

\subsection{並行平板コンデンサに働く力(p.)}
{\bf ・仮想変位}\\
\pic{./files/4/imgs/dummy.png}{0.2}
$E$が変わらないくらいわずかに狭くなったと仮定する.\\
狭くなった分だけ空間に蓄えられるエネルギーは減少する.

エネルギーの減少量$\Delta W$は
\begin{flalign}
&\Delta W=\dfrac {1}{2}\varepsilon _{0}E^{2}\times \left( S\times \Delta x\right)
\end{flalign}

電界が仕事をしたため,エネルギーは減少したと考えて
\begin{flalign}
&\Delta W=F\times \Delta x
\end{flalign}

上二つの式より
\begin{flalign}
&F\times \Delta x=\dfrac {1}{2}\varepsilon _{0}E^{2}\times S\times \Delta S\\
&F=\dfrac {1}{2}\varepsilon _{0}E^{2}S\left[ N\right]
\end{flalign}

あるいは力の密度を$P\left[ N/m^{2}\right]$と置いて
\begin{eqnarray}
P&=&\dfrac {F}{s}\\
&=&\dfrac {1}{2}\varepsilon _{0}E^{2}\left[ N/m^{2}\right]
\end{eqnarray}

ところで
\begin{flalign}
&\dfrac {\Delta W}{\Delta x}=F\\
&\dfrac {\partial W}{\partial x}=F
\end{flalign}

通常,吸引力はマイナスに取るため
\begin{flalign}
&F=-\dfrac {\partial x}{\partial x}\left[ N\right]
\end{flalign}

\begin{flalign}
&W=\dfrac {1}{2}\varepsilon _{0}E^{2}\left( Sx\right) \\
&F=-\dfrac {\partial W}{\partial x}\\
&F=\dfrac {1}{2}\varepsilon _{0}E^{2}S
\end{flalign}