\documentclass[autodetect-engine,dvipdfmx-if-dvi,ja=standard]{bxjsarticle}

% 二段組にするとき
% \documentclass[twocolumn,autodetect-engine,dvipdfmx-if-dvi,ja=standard]{bxjsarticle}

\usepackage{graphicx}        %図を表示するのに必要
\usepackage{color}           %jpgなどを表示するのに必要
\usepackage{amsmath,amssymb} %数学記号を出すのに必要
\usepackage{setspace}
\usepackage{eclclass}
\usepackage{cases}
\usepackage{here}
\usepackage{fancyhdr}
\usepackage{ascmac}

% 文書全体のスタイルを設定(主に余白)
\setlength{\topmargin}{-0.3in}
\setlength{\oddsidemargin}{0pt}
\setlength{\evensidemargin}{0pt}
\setlength{\textheight}{44\baselineskip}

% 行頭の字下げをしない
\parindent = 0pt

% ヘッダとフッタの設定
\lhead{\leftmark}  % 章(section)
\chead{}
\rhead{\rightmark} % 節(subsection)
\lfoot{}
\cfoot{-\thepage-} % ページ数
\rfoot{}

% 式の番号を(senction_num.num)のようにする
\makeatletter
\@addtoreset{equation}{section}
\def\theequation{\thesection.\arabic{equation}}
\makeatother

% 画像の貼り付けを簡単にする
\newcommand{\pic}[2]
{
  \begin{figure}[H]
    \begin{center}
      \includegraphics[scale=#2]{#1}
    \end{center}
  \end{figure}
}

% 単位の記述を簡単にする
\newcommand{\unit}[1]
{
  \, [\mathrm{#1}]
}

\title{電気磁気学}
\date{\today}

\begin{document}

\maketitle
\pagestyle{fancy}

\begin{abstract}
一関高専電気情報工学科の電気磁気学I,電気磁気学II,電気磁気学IIIの講義をまとめたもの\\
構成を少し変えてるため,授業の板書と一致しない
\end{abstract}

% 電荷と電界
\section{電荷と電界}
\subsection{電荷}
・帯電 :電気を帯びる,もつ\\
・帯電帯:電気を帯た物体\\
・電荷 :電気の実態\\
・電荷量:電荷の大きさ$[C]=[A \cdot s]$(クーロン)\\
・電荷は正,負の二種.\\
 →同種は反発し合い,異種は引き合う.\\
・帯電していない→正負の電荷を等量持っている.\\
 →電気的中性\\
・陽子は電子の1800倍の質量\\

\subsection{物質の電気的性質}
・導体:電荷をよく通す\\
・不導体:電荷をよく保持する\\
・半導体:上二つの中間的性質をもつ\\

\subsection{静電誘導}
\pic{./files/1/imgs/1.pdf}
\subsection{クーロンの法則}
\pic{./files/1/imgs/2.pdf}
点電荷:十分に小さく,点とみなせるような帯電帯\\

\begin{itembox}[l]{クーロンの法則}
  \begin{eqnarray}
    F &=&\frac {1}{4\pi \epsilon _{0}}\cdot \frac {\left| Q_{1}\right| \cdot \left| Q_{2}\right| }{r^{2}}\, [\textrm{N}] \\
    &\fallingdotseq& 9\times 10^{9} \cdot \frac {\left| Q_{1}\right| \cdot \left| Q_{2}\right| }{r^{2}}\, [\textrm{N}]\\
    \epsilon_0 &=& \mbox{真空の誘電率}\, [\textrm{F/m}]
  \end{eqnarray}
\end{itembox}

\subsection{電界}
\pic{./files/1/imgs/3.pdf}
はじめに$Q_0$の電荷を置く\\
 →周辺の空間が電気的に歪む\\

はじめに$Q$の電荷を置くと力が働く\\
 →$Q_0$によって歪んだ空間に$Q$を置くと力が働く.\\

電荷を置くことで生じる空間の電気的な歪み.
\begin{eqnarray}
  F &=&\frac {1}{4\pi \epsilon _{0}}\cdot \frac {\left| Q_{0}\right| \cdot \left| Q\right| }{r^{2}}\, [\textrm{N}] \\
  &=&\frac {\left| Q_{0}\right|}{4\pi \epsilon _{0}r^{2}}\cdot \left| Q\right|\, [\textrm{N}]
\end{eqnarray}

ここで$F=E\cdot|Q|$と置くと
\begin{eqnarray}
  E &=&\frac {\left| Q_{0}\right|}{4\pi \epsilon _{0}r^{2}}\, [\textrm{N/C}]
\end{eqnarray}

電界はベクトル量であるから

\begin{itembox}[l]{電界(電場)}
  \begin{flalign}
    \overrightarrow {F}=Q\cdot \overrightarrow {E}\, [\textrm{N/C}]
  \end{flalign}
  この関係はどのような電界でも使える
\end{itembox}

一般に,電界が存在するとき,$+1C$の点電荷を置いたなら,\\
電界の大きさ:力の大きさ\\
電界の向き :力の向き\\

\subsection{電気力線}
\pic{./files/1/imgs/4.pdf}
その曲線のある点における接線の向きが電界の向きと一致するように書いた曲線.\\
電界の大きさは電気力線の密度で表す.

\subsection{電気力線の密度と電界}
\begin{flalign}
  \mbox{電気力線の密度} = \frac{\mbox{その板を通る電気力線の数}\, \unit{\mbox{本}}}{\mbox{仮想的な板の面積}\, \unit{m^2}}
\end{flalign}

\begin{flalign}
  \mbox{電界の大きさ}\unit{N/C} = \mbox{電気力線の密度} \unit{\mbox{本}/m^2}
\end{flalign}
\pic{./files/1/imgs/5.pdf}
\begin{flalign}
&\frac {Q}{4\pi \varepsilon _{0}r^{2}}=\frac {x}{4\pi r^{2}}\\
&x=\frac {Q}{\varepsilon _{0}}
\end{flalign}

\subsection{電束と電束密度}
{\bf 電束}\\
 形状は電気力線と同じ.\\
 本数は電気力線の$\epsilon_0$倍.\\

{\bf $+Q\unit{c}$の点電荷から出る本数}\\
 電気力線:$Q/\epsilon_0\unit{\mbox{本}}$\\
 電束:$Q\unit{\mbox{本}}$\\

\begin{itembox}[l]{電束密度}
  \begin{flalign}
    \overrightarrow {D}=\varepsilon _{0}\overrightarrow {E}\unit{C/m^2}
  \end{flalign}
\end{itembox}


\subsection{ガウスの法則}
\begin{itembox}[l]{ガウスの法則}
  ある閉曲面を通って外に出る電気力線の総数はその閉曲面に含まれる電荷の総量の$Q\unit{C}$を$\epsilon_0$で割ったものに等しい.\\
   ↓\\
  ある閉曲面上で電界を面積分↓ものはその閉曲面上に含まれる電荷の総量$Q\unit{C}$を$\epsilon_0$で割ったものに等しい.
\end{itembox}

{\bf 面積分}
\pic{./files/1/imgs/6.pdf}
{\bf 電気力線の本数}
\begin{flalign}
E_{1}\times \Delta \varepsilon _{1}+E_{2}\times \Delta S_{2}+\ldots \unit{\mbox{本}/m^2}\times\unit{m^2}
\end{flalign}

{\bf 電界の面積分}
\begin{flalign}
&E_{1}\times \Delta \varepsilon _{1}+E_{2}\times \Delta S_{2}+\ldots \unit{N/C}\times\unit{m^2}\\
&=\oint _{S_{0}}EdS
\end{flalign}

\subsection{ガウスの法則を用いた計算}
\subsubsection{点電荷}
\pic{./files/1/imgs/7.pdf}
点電荷を中心とする半径$r\unit{m}$の球面上の電界の大きさを$E$とする.\\
ガウスの法則により
\begin{flalign}
&\oint_{S_{0}}EdS=\dfrac {Q}{\varepsilon _{0}}\\
&E\times\Delta S_{1}+E\times\Delta S_{2}+\ldots E\times\Delta S_{n}=\dfrac {Q}{\varepsilon _{0}}\\
&E\left( \Delta S_{1}+\Delta S_{2}+\ldots +\Delta S_{n}\right) =\dfrac {Q}{\varepsilon _{0}}\\
&E\cdot 4\pi r^{2}=\dfrac {Q}{\varepsilon _{0}}\\
&\therefore E=\dfrac {Q}{4\pi \varepsilon _{0}r^{2}}\left[ N/C\right]
\end{flalign}

\subsubsection{無限長の線電荷}
\pic{./files/1/imgs/8.pdf}
線密度$\lambda \left[ C/m\right]$の無限長の線電荷が作る電界を求める.\\
線電荷と中心軸が一致する半径が$r\left[ m\right]$,長さ$l\left[ m\right]$の円筒の側面上の電界をEとする.\\

ガウスの法則より
\begin{flalign}
&E\times 2\pi rl+0\times \pi r^{2}+0\times \pi r^{2}=\dfrac {\lambda l}{\varepsilon _{0}}\\
&E_{1}\Delta S_{1}+E_{2}\Delta S_{2}+E_{3}\Delta S_{3}+E_{4}\Delta S_{4}+E_{5}\Delta S_{5} = \dfrac {\lambda l}{\varepsilon _{0}}\\
&E\left( \Delta S_{1}+\Delta S_{2}+\Delta S_{3}\right) +0x\Delta S_{4}+0x\Delta S_{5}= \dfrac {\lambda l}{\varepsilon _{0}}\\
&E\cdot l\cdot 2\pi r=\dfrac {\lambda l}{\varepsilon _{0}}\\
&E\cdot 2\pi rl=\dfrac {\lambda l}{\varepsilon _{0}}\\
&\therefore E=\dfrac {\lambda }{2\pi \varepsilon _{0}r}\left[ N/C\right]
\end{flalign}

\subsubsection{一様に帯電した無限長の円柱}
\pic{./files/1/imgs/9.pdf}
\pic{./files/1/imgs/10.pdf}
単位長さあたり$\lambda \left[ C/m\right]$の電荷が一様に分布した円柱の内部と外部の電界の大きさを求める.\\
円柱と同じ中心軸をもつ半径が$r\left[ m\right]$,長さ$l\left[ m\right]$の円筒の側面上の電界をEとする.\\

{\bf ・外部$( r > a)$}\\
ガウスの法則より
\begin{flalign}
&\oint _{S_{0}}EdS=\dfrac {\lambda l}{\varepsilon _{0}}\\
&E\times 2\pi rl+0\times \Delta S_{n}=\dfrac {\lambda l}{\varepsilon _{0}}\\
&E=\dfrac {\lambda }{2\pi r\varepsilon _{0}}
\end{flalign}

{\bf ・内部$(0\leqq r\leqq a)$}\\
\begin{table}[htb]
\begin{center}
  \begin{tabular}{|l|c|c|} \hline
    & a & r \\ \hline
    電荷 & $\lambda l$ & $\lambda l\dfrac {r^{2}}{a^{2}}$\\
    体積 & $\pi a^{2}l$ & $\pi r^{2}l$\\
    電荷/体積 & $\dfrac {\lambda }{\pi a^{2}}$ & $\dfrac {\lambda }{\pi a^{2}}$\\ \hline
  \end{tabular}
\end{center}
\end{table}
\begin{flalign}
\dfrac {\lambda l}{\pi a^{2}l}\times \pi r^{2}l=\lambda l\dfrac {r^2 }{a^{2}}\left[ C\right]
\end{flalign}
ガウスの法則より
\begin{flalign}
&\oint _{s_{0}}EdS = \dfrac {\lambda l+\dfrac {r^{2}}{a^{2}}}{\varepsilon _{0}}\\
&E\cdot 2\pi rl+0\cdot \pi r^{2}+0\cdot \pi r^{2} = \dfrac {\lambda lr^{2}}{\varepsilon _{0}a^{2}}\\
&E=\dfrac {\lambda lr^{2}}{2\pi rl\cdot \varepsilon _{0}a^{2}}\\
&E=\dfrac {\lambda }{2\pi \varepsilon _{0}a^{2}}r
\end{flalign}
\pic{./files/1/imgs/11.pdf}

\subsubsection{一様に帯電した球}
\pic{./files/1/imgs/12.pdf}
半径$a[m]$の球に$Q[C]$の電荷が一様に分布している球と同じ中心をもつ半径$r[m]$の球面上の電界の大きさを$E$とする.\\
{\bf ・外部$(a < r)$}\\
ガウスの法則より
\begin{flalign}
&\oint_{S_{0}}EdS=\dfrac {Q}{\varepsilon _{0}}\\
&E\cdot 4\pi r^{2}=\dfrac {Q}{\varepsilon _{0}}\\
&\therefore E=\dfrac {Q}{4\pi \varepsilon _{0}r^{2}}\left[ N/C\right]
\end{flalign}

{\bf ・内部$(a < r)$}\\
\begin{table}[htb]
  \begin{center}
    \begin{tabular}{|l|c|c|} \hline
      & 体積 & 電荷 \\ \hline
      全体 & $\dfrac {4}{3}\pi a^{3}$ & $Q$\\
      求めるところ & $\dfrac {4}{3}\pi r^{3}$ & $Q\dfrac {r^{3}}{a^{3}}$\\ \hline
    \end{tabular}
  \end{center}
  \end{table}
\begin{eqnarray}
  \because \dfrac {Q}{\dfrac {4}{3}\pi a^{3}}\cdot \dfrac {4}{3}\pi r^{3}=Q\cdot \dfrac {r^{3}}{a^{3}}
\end{eqnarray}

ガウスの法則より
\begin{flalign}
&\oint _{S_{0}}EdS=\dfrac {Q\dfrac {r^{3}}{a^{3}}}{\varepsilon _{0}}\\
&E\cdot 4\pi r^{2}=\dfrac {Qr^{3}}{\varepsilon _{0}a^{3}}\\
&E=\dfrac {Q}{4\pi \varepsilon _{0}a^{3}}r
\end{flalign}
\pic{./files/1/imgs/13.pdf}

\subsubsection{表面が一様に帯電した導体球}
\pic{./files/1/imgs/14.pdf}
半径$a[m]$の導体球の表面上に$Q[C]$の電荷が一様に分布している球\\
導体球と同じ中心をもつ半径$r[m]$の球面上の電界の大きさを$E$とする.\\

{\bf ・外部$( r > a)$}\\
ガウスの法則より
\begin{flalign}
&\oint_{S_{0}}EdS=\dfrac {Q}{\varepsilon _{0}}\\
&E\cdot 4\pi r^{2}=\dfrac {Q}{\varepsilon _{0}}\\
&\therefore E=\dfrac {Q}{4\pi \varepsilon _{0}r^{2}}\left[ N/C\right]
\end{flalign}

{\bf ・内部$(0\leqq r <a)$}\\
\begin{flalign}
&\oint _{s_{0}}EdS=\dfrac {0}{\varepsilon _{0}}\\
&E \times 4\pi r^{2}=\dfrac {0}{\varepsilon _{0}}=0\\
&\therefore E=O\left[ N/C\right]
\end{flalign}
\pic{./files/1/imgs/15.pdf}

\subsubsection{一様に帯電した誘電体平面}
\pic{./files/1/imgs/16.pdf}
単位面積当たりの電荷量を$\sigma \left[ C/m^2\right]$とする.
ガウスの法則より
\begin{flalign}
&\oint _{s_{0}}EdS=\dfrac {Q_{1}}{\varepsilon _{0}}\\
&E\times S_{1}\times 2+0\times S_{2}=\dfrac {\sigma S_{1}}{\varepsilon _{0}}\\
&2ES_{1}=\dfrac {\sigma S_{1}}{\varepsilon _{0}}\\
&E=\dfrac {\sigma }{2\varepsilon _{0}}\left[ N/C\right]
\end{flalign}

\subsubsection{一様に帯電した無限大の導体平面}
\pic{./files/1/imgs/17.pdf}
単位面積当たりの電荷量を$\sigma \left[ C/m^2\right]$とする.
\begin{flalign}
&\oint _{s_{0}}EdS=\dfrac {\sigma S_{1}}{\varepsilon _{0}}\\
&E\times S_{1}+0\times S_{2}+0\times S_{2}=\dfrac {\sigma S_{1}}{\varepsilon _{0}}\\
&\therefore E=\dfrac {\sigma }{\varepsilon _{0}}\left[ N/C\right]
\end{flalign}

\subsubsection{一様に帯電した導体表面上の電界}
\pic{./files/1/imgs/18.pdf}
導体表面における電界は表面に対して垂直である.\\
なぜなら,水平方向の成分があると表面に永久電流が流れてしまうから.\\

\begin{flalign}
&ExS_{1}+0\times S_{2}+0\times S_{1}=\dfrac {\sigma S_{1}}{\varepsilon _{0}}\\
&\therefore E=\dfrac {\sigma }{\varepsilon _{0}}
\end{flalign}

{\bf 応用例}\\
半径$a[m]$,電荷量$Q[C]$を持った球の表面上の電界の大きさ\\
\pic{./files/1/imgs/19.pdf}

\begin{eqnarray}
E&=&\dfrac {\sigma }{\varepsilon _{0}}\\
&=&\dfrac {\dfrac {Q}{4\pi \varepsilon a^{2}}}{\varepsilon _{0}}\\
&=&\dfrac {Q}{4\pi \varepsilon _{0}a^{2}}
\end{eqnarray}
\pic{./files/1/imgs/20.pdf}

\newpage

% 電位
\section{電位}
\subsection{電界中で電荷を移動
するのに要する仕事}
\pic{./files/2/imgs/1.pdf}{0.2}
力$F[N]$が物体にした仕事$W[J]$は
\begin{flalign}
&W = Fl[J]
\end{flalign}\\
ここで移動距離を細かく分けてみる
\begin{eqnarray}
W &=&F(\Delta l+\Delta l+\Delta l+\ldots+\Delta l)\\
&=&F\Delta l+F\Delta l+F\Delta l+\ldots+F\Delta l
\end{eqnarray}

\subsection{電荷を運ぶのに要する仕事}
\subsubsection{ケース1}
\pic{./files/2/imgs/2.pdf}{0.2}
$Q_1$の作る電界が$Q_2$に対して仕事をした.\\
 →仕事の符号は「負」となる.\\

\subsubsection{ケース2}
\pic{./files/2/imgs/3.pdf}{0.2}
外部から$Q_2$に対して仕事をした.\\
 →仕事の符号は「正」となる.\\

\subsection{仕事の計算}
\subsubsection{ケース1}
\pic{./files/2/imgs/4.pdf}{0.23}
\begin{flalign}
&F=\dfrac {1}{4\pi \varepsilon _{0}}\times \dfrac {Q_{1}Q_{2}}{r^{2}}
\end{flalign}\\
$Q_2$の電荷が$r = r_a$から$r=r_b$まで動くとき,$Q_1$による電界が$Q_2$にした仕事$W$の近似値は
\begin{eqnarray}
W&=&-F_{1}\times \Delta r_{1}-F_{2}\times \Delta r_{2}\ldots \\
&=&\sum ^{3}_{k=1}F_{k}\Delta r_{k}
\end{eqnarray}\\
分割を限りなく細かくすると,積分を用いて$W$を正確に求めることができる.
\begin{flalign}
&W=-\int ^{r_{b}}_{r_{a}}Fdr\left[ J\right]
\end{flalign}

\subsubsection{ケース2}
\pic{./files/2/imgs/5.pdf}{0.2}
クーロン力に逆らって,外部から$Q_2$にした仕事$W$は
\begin{eqnarray}
W&=&\int ^{r_{c}}_{r_{b}}Fdr\left[ J\right] \\
&=&-\int ^{r_{b}}_{r_{c}}Fdr\left[ J\right]
\end{eqnarray}\\

\subsubsection{まとめ}
2つのケースのいずれの場合でも,以下のように表せる.
\begin{flalign}
&W=-\int ^{\mbox{終点}}_{\mbox{始点}}(\mbox{クーロン力})dr\left[ J\right]\\
&\begin{cases}W <0:\mbox{電界がした力}\\
W >0:\mbox{外部がした力}\end{cases}
\end{flalign}

\subsection{電位}
\subsubsection{無限遠点から電荷を運ぶのに要する仕事}
\pic{./files/2/imgs/6.pdf}{0.2}
\begin{eqnarray}
W&=&-\int ^{r_{a}}_{\infty }\dfrac {Q_{1}Q_{2}}{4\pi \varepsilon _{0}r^{2}}dr\\
&=&\dfrac {Q_{1}Q_{2}}{4\pi \varepsilon _{0}}\left[ r^{-1}\right] ^{r_{a}}_{\infty }\\
&=&\dfrac {Q_{1}Q_{2}}{4\pi \varepsilon _{0}}\left( \dfrac {1}{r_{a}}-0\right) \\
&=&\dfrac {Q_{1}Q_{2}}{4\pi \varepsilon _{0}r_{a}}\left[ J\right]
\end{eqnarray}\\

\subsubsection{無限遠点から$+1C$の電荷を運ぶのに要する仕事}
前問において,$Q_2 = 1[C]$とすればよい.
\begin{eqnarray}
\therefore W&=&\dfrac {Q_{1}}{4\pi \varepsilon _{0}r_{a}}\left[ J\right] \\
&=&-\int ^{r_{a}}_{\infty }\dfrac {Q_{1}Q_{2}}{4\pi \varepsilon _{0}r^{2}}dr\\
&=&-\int ^{r_{a}}_{\infty }EQ_{2}dr
\end{eqnarray}\\

ここで$Q_2 = 1[C]$だから
\begin{flalign}
&W=-\int ^{r_{a}}_{\infty }Edr
\end{flalign}

\subsubsection{電位}
\begin{itembox}[l]{電位}
無限遠点から$+1C$の電荷を運ぶのに要する仕事を$r=r_a$における電位と定義する.\\
単位は$[V]$を使用する.
\begin{flalign}
&V_a=-\int ^{r_{a}}_{\infty }Edr\left[ V\right]
\end{flalign}
\end{itembox}\\

\subsubsection{電位の基準}
1. 電気磁気学:無限遠点\\
2. 電力   :大地\\
3. 回路   :任意

\subsubsection{点電荷による電界}
\pic{./files/2/imgs/7.pdf}{0.1}
点$P$の電位は
\begin{flalign}
&V_{a}=\dfrac {Q}{4\pi \varepsilon _{0}r_{a}}\left[ V\right]
\end{flalign}

\subsection{電位差}
\subsubsection{電位差}
\pic{./files/2/imgs/8.pdf}{0.4}
$V_a-V_b$を点$B$に対する点$A$の電位差という.\\
 →点$B$から点$A$に$+1C$を運ぶのに要する仕事.\\
電位差を電圧ともいう.

\subsubsection{計算方法}
点$B$に対する点$A$の電位差$V_{AB}$は\\
\begin{flalign}
&V_{AB}=-\int ^{r_{a}}_{r_b}(\mbox{電界})dr
\end{flalign}\\

原点に$Q[C]$がある場合は\\
\begin{eqnarray}
V_{AB}&=&-\int ^{r_{a}}_{r_{b}}\dfrac {Q}{4\pi \varepsilon _{0}}dr\\
&=&\dfrac {Q}{4\pi \varepsilon _{0}}\left[ r^{-1}\right] ^{r_{a}}_{r_{b}}\\
&=&\dfrac {Q}{4\pi \varepsilon _{0}}\left( \dfrac {1}{r_{a}}-\dfrac {1}{r_{b}}\right) \left[ V\right]
\end{eqnarray}

\subsubsection{等電位線と電気力線の関係}
\pic{./files/2/imgs/9.pdf}{0.4}
・電界中で,電位$V$の等しい点を連ねて作った仮想的な線を等電位線という.\\
・二次元の時は等電位線,三次元の時は,等電位面.\\
・等電位線と電気力線は垂直に交わる.\\
・電位の山を下る時,最も急な傾斜が電界の向き.\\
・等電位線に沿って電荷を動かした時,仕事はしない.\\
・動かす方向に力はない.

\subsection{電位の傾き}
\subsubsection{勾配}
\pic{./files/2/imgs/10.pdf}{0.15}
\begin{flalign}
&\mbox{平均変化率}=\dfrac {\Delta y}{\Delta x}\\
&\mbox{勾配}=\dfrac {dy}{dx}\left[ (\mbox{yの単位})/m\right]
\end{flalign}


\subsubsection{電位の傾き}
\pic{./files/2/imgs/11.pdf}{0.2}
$Q$による電界$E$が行った仕事$\Delta W$は
\begin{eqnarray}
\Delta W&=&-(\mbox{力})\times (\mbox{距離})\\
&=&-qE\times \Delta x
\end{eqnarray}\\

この時,$E$が一定とみなせるほど$\Delta x$は小さい.\\
$q=1C$なら,$\Delta W$は電位差$\Delta V$と置き換えられる.
\begin{flalign}
&\Delta V=-E \times \Delta x\\
&\dfrac {\Delta V}{\Delta x}=-E\\
&E=-\dfrac {\Delta V}{\Delta x}\left[ V/m\right]
\end{flalign}

$\Delta x$が限りなく小さければ
\begin{flalign}
&E=-\dfrac {dV}{dx}\left[ V/m\right]\left[ N/C\right]
\end{flalign}\\

すなわち,電界は電位の勾配にマイナスをつけたもの.\\
 →$[N/C]=[V/m]$

\newpage

% 様々な帯電体による電界,電位
\section{様々な帯電体による電界,電位(p.78)}
\subsection{一様に帯電した球(p.82)}
\subsubsection{電界}
\pic{./files/1/imgs/12.pdf}{0.2}
\ref{ChargedBall}と同様に,半径$a[m]$の球に$Q[C]$の電荷が一様に分布している球と同じ中心をもつ半径$r[m]$の球面上の電界の大きさを$E$とする.\\
{\bf ・外部$(a < r)$}\\
ガウスの法則より
\begin{flalign}
&\oint_{S_{0}}EdS=\dfrac {Q}{\varepsilon _{0}}\\
&E\cdot 4\pi r^{2}=\dfrac {Q}{\varepsilon _{0}}\\
&\therefore E=\dfrac {Q}{4\pi \varepsilon _{0}r^{2}}\left[ V/m\right]
\end{flalign}

{\bf ・内部$(a < r)$}\\
\begin{table}[htb]
  \begin{center}
    \begin{tabular}{|l|c|c|} \hline
      & 体積 & 電荷 \\ \hline
      全体 & $\dfrac {4}{3}\pi a^{3}$ & $Q$\\
      求めるところ & $\dfrac {4}{3}\pi r^{3}$ & $Q\dfrac {r^{3}}{a^{3}}$\\ \hline
    \end{tabular}
  \end{center}
\end{table}
\begin{eqnarray}
  \because \dfrac {Q}{\dfrac {4}{3}\pi a^{3}}\cdot \dfrac {4}{3}\pi r^{3}=Q\cdot \dfrac {r^{3}}{a^{3}}
\end{eqnarray}

ガウスの法則より
\begin{flalign}
&\oint _{S_{0}}EdS=\dfrac {Q\dfrac {r^{3}}{a^{3}}}{\varepsilon _{0}}\\
&E\cdot 4\pi r^{2}=\dfrac {Qr^{3}}{\varepsilon _{0}a^{3}}\\
&E=\dfrac {Q}{4\pi \varepsilon _{0}a^{3}}r
\end{flalign}
\pic{./files/1/imgs/13.pdf}{0.2}

\subsubsection{電位}
誘電体球の中心から$R[m]$離れた点の電位を$V_R$とする.\\
{\bf ・外部$(a < r)$}\\
\begin{eqnarray}
V_{R}&=&-\int ^{R}_{\infty }Edr\\
&=&-\int ^{R}_{\infty }\dfrac {Q}{4\pi \varepsilon _{0}r^{2}}dr\\
&=&\dfrac {Q}{4\pi \varepsilon _{0}}\int ^{R}_{\infty }\left( -r^{2}\right) dr\\
&=&\dfrac {Q}{4\pi \varepsilon _{0}}\left[ r^{-1}\right] ^{R}_{\infty }\\
&=&\dfrac {Q}{4\pi \varepsilon _{0}}\left( \dfrac {1}{R}-0\right) \\
&=&\dfrac {Q}{4\pi \varepsilon _{0}R}\left[ V\right]
\end{eqnarray}

{\bf ・内部$(a < r)$}\\
\begin{eqnarray}
  V_{R}&=&-\int ^{R}_{\infty }Edr\\
  &=&-\left( \int ^{a}_{\infty }Edr+\int ^{R}_{a}Edr\right) \\
  &=&\dfrac {Q}{4\pi \varepsilon _{0}a}-\dfrac {Q}{4\pi \varepsilon _{0}a^{3}}\times \int ^{R}_{a}rdr\\
  &=&\dfrac {Q}{4\pi \varepsilon _{0}a}-\dfrac {Q}{4\pi \varepsilon _{0}a^{3}}\times \left[ \dfrac {1}{2}r^{2}\right] ^{R}_{a}dr\\
  &=&\dfrac {Q}{4\pi \varepsilon _{0}a}-\dfrac {Q}{4\pi \varepsilon _{0}a^{3}}\times \dfrac {1}{2}\left( R^{2}-a^{2}\right)\\
  &=&\dfrac {Q}{4\pi \varepsilon _{0}a}-\dfrac {Q}{8\pi \varepsilon _{0}a^{3}}R^{2}+\dfrac {Q}{8\pi \varepsilon _{0}a^{3}}\\
  &=&\dfrac {Q}{4\pi \varepsilon _{0}a}-\dfrac {Q}{8\pi \varepsilon _{0}a^{3}}R^{2}+\dfrac {Q}{8\pi \varepsilon _{0}a^{3}}\\
  &=&\dfrac {Q}{4\pi \varepsilon _{0}a}\left( 1-\dfrac {R^{2}}{2a^{2}}+\dfrac {1}{2}\right)\\
  &=&\dfrac {Q}{4\pi \varepsilon _{0}a}\left( \dfrac {3}{2}-\dfrac {R^{2}}{2a^{2}}\right) \left[ V\right]
\end{eqnarray}

Rについて,\\
内部の電位は上に凸の放物線($R=0$の時に最大値をとる)\\
外部の電位は分数関数\\
したがって,グラフは
\pic{./files/3/imgs/1.pdf}{0.2}

\subsection{表面が一様に帯電した導体球(p.87)}
\subsubsection{電界}
\pic{./files/1/imgs/14.pdf}{0.2}
半径$a[m]$の導体球の表面上に$Q[C]$の電荷が一様に分布している球\\
導体球と同じ中心をもつ半径$r[m]$の球面上の電界の大きさを$E$とする.\\

{\bf ・外部$( r > a)$}\\
ガウスの法則より
\begin{flalign}
&\oint_{S_{0}}EdS=\dfrac {Q}{\varepsilon _{0}}\\
&E\cdot 4\pi r^{2}=\dfrac {Q}{\varepsilon _{0}}\\
&\therefore E=\dfrac {Q}{4\pi \varepsilon _{0}r^{2}}\left[ V/m\right]
\end{flalign}

{\bf ・表面$(r = a)$}\\
\begin{eqnarray}
E&=&\dfrac {\sigma }{\varepsilon _{0}}=\dfrac {\dfrac {Q}{4\pi a^{2}}}{\varepsilon _{0}}\\
 &=&\dfrac {Q}{4\pi \varepsilon _{0}a^{2}}\left[ V/m\right]
\end{eqnarray}

{\bf ・内部$(0\leqq r <a)$}\\
\begin{flalign}
&\oint _{s_{0}}EdS=\dfrac {0}{\varepsilon _{0}}\\
&E \times 4\pi r^{2}=\dfrac {0}{\varepsilon _{0}}=0\\
&\therefore E=0\left[ V/m\right]
\end{flalign}

外部と表面について,違う方法で求めたが,結果が同じになったので,同じ関数で表せるものとして扱う.\\
電界のグラフは,
\pic{./files/1/imgs/15.pdf}{0.2}

\subsubsection{電位}
{\bf ・表面,外部$(a \leqq r)$}\\
\begin{eqnarray}
V_{R}&=&-\int ^{R}_{\infty }Edr\\
&=&-\int ^{R}_{\infty }\dfrac {Q}{4\pi \varepsilon _{0}r^{2}}dr\\
&=&\dfrac {Q}{4\pi \varepsilon _{0}R}\left[ V\right]
\end{eqnarray}

{\bf ・内部$(0\leqq r <a)$}\\
\begin{eqnarray}
  V_{R}&=&-\int ^{R}_{\infty }Edr\\
  &=&-\left( \int ^{a}_{\infty }Edr+\int ^{R}_{a}Edr\right) \\
  &=&\dfrac {Q}{4\pi \varepsilon _{0}a}- \int ^{R}_{a}0dr\\
  &=&\dfrac {Q}{4\pi \varepsilon _{0}a}\left[ V\right]
\end{eqnarray}

電位のグラフは,
\pic{./files/3/imgs/2.pdf}{0.2}

{\bf $R$が$0$から離れていく時,電位が大きくなることはない}\\

このような問題を解くときのポイント\\
1.電荷の分布を決める\\
2.電界を内側から決める\\
3.電位は必ず無限遠点から決める\\

\subsection{電気双極子(p.78)}
\subsubsection{電気双極子}
\pic{./files/3/imgs/3.pdf}{0.25}
電気双極子:大きさが等しく,符号が逆の極めて接近して存在するもの.

\subsubsection{電気双極子により作られる電位}
\pic{./files/3/imgs/4.pdf}{0.3}
電位は各点電荷による電位の和をとれば良い\\
$+Q$,$-Q$の点電荷による電位$V_p$は
\begin{eqnarray}
  V_p &=& \dfrac{Q}{4\pi \varepsilon _{0}\overline{AP}} + \dfrac{-Q}{4\pi \varepsilon _{0}\overline{BP}}\\
  &=& \dfrac{Q}{4\pi \varepsilon _{0}\overline{AP}} - \dfrac{Q}{4\pi \varepsilon _{0}\overline{BP}}\\
  &=& \dfrac{Q}{4\pi \varepsilon _{0}} \cdot \dfrac{\overline{BP}-\overline{AP}}{\overline{AP}\cdot \overline{BP}}\\
  &\fallingdotseq& \dfrac{Q}{4\pi \varepsilon _{0}} \cdot \dfrac{l \cdot \cos(\theta)}{r^2}\\
  &=& \dfrac{Ql \cos(\theta)}{4\pi \varepsilon _{0}r^2}\left[ V\right]
\end{eqnarray}

\subsubsection{双極子モーメント}
ここで,$P=Ql[Cm]$と置くと,電気双極子により作られる電位は次のようになる.
\begin{flalign}
  &V_p = \dfrac{P}{4\pi \varepsilon _{0}r^2} \cos(\theta)\left[ V\right]
\end{flalign}

この$P$を双極子モーメントと呼ばれ,次のように定義される.
大きさ:$P=Ql[Cm]$\\
向き:負から正の向き\\

\newpage

% 静電容量
\section{静電容量(p.99)}
\subsection{静電容量(p.102)}
\subsubsection{ケース1}
\pic{./files/4/imgs/dummy.png}{0.2}
導体に$Q[C]$の電荷を与えたとき,電圧$V[V]$
\begin{flalign}
  &C=\dfrac{Q}{V} \left[ F\right]
\end{flalign}
このときの$C$を静電容量という.

\subsubsection{ケース2}
\pic{./files/4/imgs/dummy.png}{0.2}
導体A,Bにそれぞれ$+Q[C]$,$-Q[C]$の電荷を与えたとき,Bに対するAの電位差を$V_{AB}$とすると
\begin{flalign}
  &C=\dfrac{Q}{V_{AB}} \left[ F\right]
\end{flalign}
これをAB間の静電容量という.

\subsection{静電容量の計算(p.104)}
\subsubsection{導体球の静電容量(p.104)}
\pic{./files/4/imgs/dummy.png}{0.2}
{\bf ・電界}\\
{\bf ・外部$(a < r)$}\\
\begin{flalign}
  &E=\dfrac {Q}{4\pi \varepsilon _{0}r^{2}}\left[ V/m\right]
\end{flalign}
{\bf ・表面$(a = r)$}\\
\begin{flalign}
&E=\dfrac {Q}{4\pi \varepsilon _{0}a^{2}}\left[ V/m\right]
\end{flalign}

{\bf ・電位}\\
\begin{eqnarray}
V&=&-\int ^{a}_{\infty }Edr\\
&=&-\int ^{a}_{\infty }\dfrac {Q}{4\pi \varepsilon _{0}r^{2}}dr\\
&=&\dfrac {Q}{4\pi \varepsilon _{0}}\left[ \dfrac {1}{r}\right] ^{a}_{\infty }\\
&=&\dfrac {Q}{4\pi \varepsilon _{0}a}\left[ V\right]
\end{eqnarray}

{\bf ・静電容量}\\
\begin{eqnarray}
C&=&\dfrac {Q}{V}\\
&=&4\pi \varepsilon _{0}a\left[ F\right]
\end{eqnarray}

\subsubsection{同心球間の静電容量(p.104)}
\pic{./files/4/imgs/dummy.png}{0.2}
{\bf ・電界}\\
\begin{flalign}
&E\times 4\pi r^{2}=\dfrac {Q}{\varepsilon _{0}}\\
&E=\dfrac {Q}{4\pi \varepsilon _{0}r^{2}}\left[ V/m\right]
\end{flalign}

{\bf ・電位差}\\
\begin{eqnarray}
V_{AB}&=&-\int ^{a}_{b}Edr\\
&=&-\int ^{a}_{b}\dfrac {Q}{4\pi \varepsilon _{0}r^{2}}dn\\
&=&\dfrac {1}{4\pi \varepsilon _{0}}\left[ \dfrac {1}{r}\right] ^{a}_{b}\\
&=&\dfrac {1}{4\pi \varepsilon _{0}}\left( \dfrac {1}{a}-\dfrac {1}{b}\right)\\
&=&\dfrac {Q\left( b-a\right) }{4\pi \varepsilon _{0}ab}\left[ V\right]
\end{eqnarray}

{\bf ・静電容量}\\
\begin{eqnarray}
C&=&\dfrac {Q}{V_{ab}}\\
&=&\dfrac {4\pi \varepsilon _{0}ab}{b-a}\left[ F\right]
\end{eqnarray}

\subsubsection{同心円筒間の静電容量(p.106)}
\pic{./files/4/imgs/dummy.png}{0.2}
無限長の同軸ケーブルのイメージ\\
円柱A:$+\lambda [C/m]$\\
円柱B:$-\lambda [C/m]$\\

{\bf ・電界}\\
\begin{flalign}
&E\times 2\pi rl+0\times \pi r^{2}\times 2=\dfrac {\lambda l}{\varepsilon _{0}}\\
&E=\dfrac {\lambda }{2\pi \varepsilon _{0}r}\left[ V/m\right]
\end{flalign}

{\bf ・電位差}\\
\begin{eqnarray}
V_{AB}&=&-\int ^{a}_{b}Edn\\
&=&-\int ^{a}_{b}\dfrac {1}{2\pi \varepsilon _{0}r}dr\\
&=&-\dfrac {\lambda }{2\pi \varepsilon _{0}}\int ^{a}_{b}\dfrac {1}{r}dn\\
&=&-\dfrac {\lambda }{2\pi \varepsilon _{0}}\left[ \log \left| r\right| \right] ^{a}_{b}\\
&=&-\dfrac {\lambda }{2\pi \varepsilon _{0}}\left( \log a-\log b\right) \\
&=&\dfrac {\lambda }{2\pi \varepsilon _{0}}\left( \log b-\log a\right) \\
&=&\dfrac {\lambda }{2\pi \varepsilon _{0}}\log \dfrac {b}{a}\left[ V\right]
\end{eqnarray}

{\bf ・静電容量}\\
単位長さあたりの静電容量を$C\left[ F/m\right]$と置いて
\begin{flalign}
&Q=CV\\
&\lambda l=Cl\cdot V_{AB}\\
&C=\dfrac {\lambda }{V_{AB}}\\
&C=\dfrac {2\pi \varepsilon _{0}}{\log \dfrac {b}{a}}\left[ F/m\right]
\end{flalign}

\subsubsection{並行平板間の静電容量(p.107)}
\pic{./files/4/imgs/dummy.png}{0.2}
導体A:$+\lambda [C]$\\
導体B:$-\lambda [C]$\\
{\bf ・電界}\\
\begin{eqnarray}
E&=&\dfrac {\left| \sigma \right| }{\varepsilon _{0}}\\
&=&\dfrac {\left| \dfrac {Q}{S}\right| }{\varepsilon _{0}}\\
&=&\dfrac {Q}{\varepsilon _{0}S}\left[ V/m\right]
\end{eqnarray}

{\bf ・電位差}\\
\begin{eqnarray}
V_{AB}&=&E\cdot d\\
&=&\dfrac {Qd}{\varepsilon _{0}S}\left[ V\right]
\end{eqnarray}

{\bf ・静電容量}\\
\begin{eqnarray}
C&=&\dfrac {Q}{V_{AB}}\\
&=&\dfrac {Q}{\dfrac {Qd}{\varepsilon _{0}S}}\\
&=&\dfrac {\varepsilon _{0}S}{d}
\end{eqnarray}

\subsubsection{無限長の並行導線間の静電容量(p.108)}
\pic{./files/4/imgs/dummy.png}{0.2}
中心間隔:$d[m]$\\
半径:$a[m]$\\

{\bf ・電界}\\
$x$軸状の電界の大きさ$E$は
\begin{flalign}
&E=\dfrac {\lambda }{2\pi \varepsilon _{0}x}+\dfrac {\lambda }{2\pi \varepsilon _{0}\left( d-x\right) }
\end{flalign}

{\bf ・電位差}\\
\begin{eqnarray}
V_{AB}&=&-\int ^{a}_{d-a}Edx\\
&=&-\dfrac {\lambda }{2\pi \varepsilon _{0}}\int ^{a}_{d-a}\left( \dfrac {1}{x}+\dfrac {1}{dx}\right) dx\\
&=&-\dfrac {\lambda }{2\pi \varepsilon _{0}}\left[ \log \left| x\right| -\log \left| d-x\right| \right] ^{a}_{d-a}\\
&=&-\dfrac {\lambda }{2\pi \varepsilon _{0}} \left( \log a-\log \left( d-a\right) \log \left( d-a\right) +\log a\right) \\
&=&\dfrac {\lambda }{2\pi \varepsilon _{0}}\times 2\left( \log \left( d-a\right) -\log a\right)\\
&=&\dfrac {\lambda }{\pi \varepsilon _{0}}\log \dfrac {d-a}{a}\left[ V\right]
\end{eqnarray}

{\bf ・静電容量}\\
単位長さあたりの静電容量を$C\left[ F/m\right]$と置いて
\begin{flalign}
&\lambda l=ClV_{AB}\\
&C=\dfrac {\lambda }{V_{AB}}\\
&C=\dfrac {\pi \varepsilon _{0}}{\log \dfrac {d-a}{a}}\left[ F/m\right]
\end{flalign}

\subsection{電気影像法(p.117)}
\pic{./files/4/imgs/dummy.png}{0.2}
静電誘導による電界,電荷分布を求める方法.

\pic{./files/4/imgs/dummy.png}{0.2}
\begin{flalign}
&E_{1}=E_{2}=\dfrac {Q}{4\pi \varepsilon _{0}r^{2}}
\end{flalign}

求める電界の大きさを$E$とおく.
\begin{eqnarray}
E&=&E_{1}\sin \theta +E_{2}\sin \theta \\
&=&2E,\sin \theta \\
&=&2\times \dfrac {Q}{4\pi \varepsilon _{0}r^{2}}\cdot \dfrac {a}{r}\\
&=&\dfrac {Qa}{2\pi \varepsilon \cdot r^{3}}\\
&=&\dfrac {Qa}{{\pi \varepsilon _{0}\left( x^{2}+a^{2}\right)}^{ \frac {3}{2}}}\left[ V/m\right]
\end{eqnarray}

ところで,導体表面の電界と電荷密度の関係(クーロンの定理)より
\begin{flalign}
&E=\dfrac {\sigma }{\varepsilon _{0}}\left[ V/m\right]\\
&\sigma =E\varepsilon _{0}
\end{flalign}

よって
\begin{eqnarray}
\sigma &=& E\varepsilon _{0}\\
&=&\dfrac {Qa}{2\pi \left( x^{2}+a^{2}\right) ^{\frac {3}{2}}}\left[ C/m^{2}\right]
\end{eqnarray}

\subsection{コンデンサに蓄えられるエネルギー(p.125)}
電位が$3V$
 →無限遠点から$+1C$の電荷を運ぶのに必要な仕事が$3[J]$

\pic{./files/4/imgs/dummy.png}{0.2}
電位が$frac{q}{C}$の導体に$\Delta q[C]$の電荷を運ぶのに必要な仕事を$\Delta W[J]$と置くと
\begin{flalign}
&\Delta W=\dfrac {q}{C}\Delta q
\end{flalign}

電荷量が$0[C]$から$Q[C]$になるまでの仕事$W$は
\begin{eqnarray}
W&=&\sum \Delta W\\
&=&\sum \dfrac {q}{C}\Delta q\\
&=&\int ^{Q}_{0}\dfrac {q}{C}dq\\
&=&\left[ \dfrac {q^{2}}{2C}\right] ^{Q}_{0}\\
&=&\dfrac {Q^{2}}{2C}\left[ J\right] \\
&=&\dfrac {Q\cdot CV}{2C}\\
&=&\dfrac {Q\cdot V}{2}\\
&=&\dfrac {1}{2}CV^{2}\left[ J\right] \\
\end{eqnarray}

導体に蓄えられるエネルギー
 →2つの導体からなるコンデンサに蓄えられるエネルギーもこの形になる.

\subsection{電界に蓄えられるエネルギーの密度(p.127)}
\pic{./files/4/imgs/dummy.png}{0.2}
並行平板コンデンサに蓄えられるエネルギー$W[J]$は
\begin{eqnarray}
W&=&\dfrac {1}{2}CV^{2}\\
&=&\dfrac {1}{2}\dfrac {\varepsilon _{0}S}{d}\times \left( Ed\right) ^{2}\\
&=&\dfrac {1}{2}\varepsilon SdE^{2}\left[ J\right]
\end{eqnarray}

電極の体積で割ったものを$w\left[ J/m^{3}\right]$と置くと
\begin{eqnarray}
w&=&\dfrac {W}{Sd}\\
&=&\dfrac {\dfrac {1}{2}\varepsilon _{0}SdE^{2}}{Sd}\\
&=&\dfrac {1}{2}\varepsilon _{0}E^{2}\left[ J/m^{3}\right]
\end{eqnarray}
これは,電界に蓄えられている電気的エネルギーの密度を表している.

\subsection{並行平板コンデンサに働く力(p.)}
{\bf ・仮想変位}\\
\pic{./files/4/imgs/dummy.png}{0.2}
$E$が変わらないくらいわずかに狭くなったと仮定する.\\
狭くなった分だけ空間に蓄えられるエネルギーは減少する.

エネルギーの減少量$\Delta W$は
\begin{flalign}
&\Delta W=\dfrac {1}{2}\varepsilon _{0}E^{2}\times \left( S\times \Delta x\right)
\end{flalign}

電界が仕事をしたため,エネルギーは減少したと考えて
\begin{flalign}
&\Delta W=F\times \Delta x
\end{flalign}

上二つの式より
\begin{flalign}
&F\times \Delta x=\dfrac {1}{2}\varepsilon _{0}E^{2}\times S\times \Delta S\\
&F=\dfrac {1}{2}\varepsilon _{0}E^{2}S\left[ N\right]
\end{flalign}

あるいは力の密度を$P\left[ N/m^{2}\right]$と置いて
\begin{eqnarray}
P&=&\dfrac {F}{s}\\
&=&\dfrac {1}{2}\varepsilon _{0}E^{2}\left[ N/m^{2}\right]
\end{eqnarray}

ところで
\begin{flalign}
&\dfrac {\Delta W}{\Delta x}=F\\
&\dfrac {\partial W}{\partial x}=F
\end{flalign}

通常,吸引力はマイナスに取るため
\begin{flalign}
&F=-\dfrac {\partial x}{\partial x}\left[ N\right]
\end{flalign}

\begin{flalign}
&W=\dfrac {1}{2}\varepsilon _{0}E^{2}\left( Sx\right) \\
&F=-\dfrac {\partial W}{\partial x}\\
&F=\dfrac {1}{2}\varepsilon _{0}E^{2}S
\end{flalign}
\newpage

% 誘電体
\begin{flalign}
&\left| \overrightarrow {P}\right| =\sigma_{p} \left[ C/m^2\right]
\end{flalign}

\begin{flalign}
&E_{v}=\dfrac {\sigma _{t}}{\varepsilon _{0}}\left[ v/m\right]
\end{flalign}

\begin{flalign}
&E_{d}=\dfrac {\sigma _{t}-\sigma _{p}}{\varepsilon _{0}}\left[ V/m\right]
\end{flalign}

\begin{eqnarray}
\varepsilon _{0}&=&\dfrac {E_{v}}{\sigma _{t}}\left( \sigma _{t}-\sigma _{p}\right)\\
=\left( 1-\dfrac {\sigma _{p}}{\sigma _{t}}\right) E_{v}
\end{eqnarray}

\begin{eqnarray}
E_{d}&=&\dfrac {\sigma _{t}}{\varepsilon _{0}}-\dfrac {\sigma _{p}}{\varepsilon _{0}}\\
&=&E_{v}-\dfrac {1}{\varepsilon _{0}}P\\
&=&E^{\rightarrow _{v}}-\dfrac {1}{\varepsilon _{0}}\overrightarrow {P}\
\end{eqnarray}


\begin{flalign}
&C_{u}=\dfrac {\varepsilon _{0}S}{l}\left[ F\right]
\end{flalign}

\begin{eqnarray}
C_{d}&=&\dfrac {\varepsilon S}{l}\\
&=&\dfrac {\varepsilon _{s}\varepsilon _{0}S}{l}\left[ F\right]
\end{eqnarray}

\begin{flalign}
&C_{d}=\varepsilon _{s}C_{v}
\end{flalign}

\begin{eqnarray}
\dfrac {E_{d}}{E_{v}}&=&\dfrac {V_{d}/l}{V_{v}/l}=\dfrac {V_{d}}{V_{v}}\\
&=&\dfrac {Q/Cd}{Q/C_{v}}=\dfrac {C_{v}}{C_{d}}
\end{eqnarray}

\begin{flalign}
&\dfrac {E_{d}}{E_{v}}=\dfrac {C_{v}}{\varepsilon _{s}C_{v}}\\
&E_{d}=\dfrac {1}{\varepsilon _{s}}\cdot E_{v}
\end{flalign}

\begin{flalign}
&D_{v}=\varepsilon _{0}E_{v}
\end{flalign}

\begin{flalign}
&D_{d}=\varepsilon E_{d}
\end{flalign}

\begin{flalign}
&E_{d}=\dfrac {1}{\varepsilon _{s}}E_{v}\\
&\varepsilon _{s}E_{d}=E_{v}\\
&\varepsilon _{s}\varepsilon _{0}E_{d}=\varepsilon _{0}E_{v}\\
&D_{d}=D_{v}\left[ C/m^{2}\right]
\end{flalign}

\begin{flalign}
&E_{v}=\dfrac {\sigma _{t}}{\varepsilon _{0}}\\
&\varepsilon _{0}E_{v}=\sigma _{t}\\
&Dv=\sigma _{t}\left[ C/m^{2}\right]\\
&D_{d}=\sigma _{t}\left[ C/m^{2}\right]
\end{flalign}

\begin{flalign}
&F=\dfrac {Q_{1}Q_{2}}{4\pi \varepsilon r^{2}}\left[ N\right]
\end{flalign}

\begin{eqnarray}
W&=&\dfrac {1}{2}\varepsilon E^{2}\\
&=&\dfrac {1}{2}ED\left[ J/m^{3}\right]
\end{eqnarray}

\begin{eqnarray}
f&=&\dfrac {1}{2}\varepsilon E^{2}\\
&=&\dfrac {1}{2}ED\left[ N/m^{2}\right]
\end{eqnarray}

\begin{flalign}
&\theta _{1}\neq \theta _{2}
\end{flalign}

\begin{flalign}
&\oint Edl=0\\
&E_{it}\Delta l-E_{2t}\Delta l=0\\
&E_{1t}=E_{2t}
\end{flalign}

\newpage

% 電流と抵抗
\begin{eqnarray}
\lim _{\Delta t\rightarrow 0}\dfrac {\Delta Q}{\Delta t}&=&\dfrac {dQ}{dt}\\
&=&I\left[ A\right]
\end{eqnarray}

\begin{flalign}
&\overrightarrow {J}=\lim _{\Delta S\rightarrow 0}\dfrac {\Delta \overrightarrow {I}}{\Delta S}\left[ A/m^{2}\right]
\end{flalign}

\begin{flalign}
&v\Delta S\left[ m^{3}\right]
\end{flalign}

\begin{flalign}
&\Delta I=env\Delta S\left[ A\right]
\end{flalign}

\begin{flalign}
&J=\dfrac {\Delta I}{\Delta S}=env\left[ A/m^{2}\right]\\
&\overrightarrow {J}=en\overrightarrow {v}\left[ A/m^{2}\right]
\end{flalign}

\begin{flalign}
&R\propto l\\
&R\propto \dfrac{1}{S}\\
&R\propto \dfrac{l}{S}
\end{flalign}

\begin{flalign}
&R=\rho \dfrac {l}{S}
\end{flalign}

\begin{flalign}
&\sigma =\dfrac {1}{\rho }\left[ S/m\right]
\end{flalign}




\newpage

% 磁界
\begin{flalign}
&\Delta H\propto I\\
&\Delta H\propto \Delta l\\
&\Delta H\propto \dfrac {1}{r^{2}}\\
&\Delta H\propto \dfrac {I\cdot \Delta l}{r^{2}}
\end{flalign}

\begin{flalign}
&\Delta H=\dfrac {1}{4\pi }\cdot \dfrac {I\cdot \Delta l}{r^{2}}
\end{flalign}

\begin{flalign}
&\Delta H=\dfrac {1}{4\pi }\dfrac {I\cdot \Delta l}{r^{2}}\sin \theta \left[ A/m\right]
\end{flalign}

\begin{flalign}
&\overrightarrow {a}\times \overrightarrow {b}
\end{flalign}

\begin{flalign}
&\left| \overrightarrow {a}\times \overrightarrow {b}\right| =\left| \overrightarrow {a}\right| \cdot \left| \overrightarrow {b}\right| \sin \theta
\end{flalign}

\begin{flalign}
&\left| \Delta \overrightarrow {l}\times \overrightarrow {r}\right| =\Delta l\cdot r\cdot \sin \theta
\end{flalign}

\begin{eqnarray}
\Delta H&=&\dfrac {1}{4\pi }\dfrac {I\cdot \Delta l}{r^{2}}\sin \theta \\
&=&\dfrac {I}{4\pi }\cdot \dfrac {\Delta lr\sin \theta }{r^{3}}\\
&=&\dfrac {I}{4\pi }\cdot \dfrac {\left| \Delta \overrightarrow {l}\times \overrightarrow {r}\right| }{r^{3}}
\end{eqnarray}

\begin{flalign}
&\Delta \overrightarrow {H}=\dfrac {I}{4\pi }\cdot \dfrac {\Delta \overrightarrow {l}\times \overrightarrow {r}}{r^{3}}\left[ A/m\right]
\end{flalign}

\begin{flalign}
&\overrightarrow {B}=\mu _{0}\overrightarrow {H}\left[ T\right] \\
&\mu _{0}=4\pi \times 10^{-7}\left[ H/m\right]
\end{flalign}

\begin{flalign}
&\Delta H=\dfrac {I\Delta x\sin \theta }{4\pi r^{2}}
\end{flalign}

\begin{flalign}
&\sin \theta =\dfrac {a}{r}\rightarrow r=\dfrac {a}{\sin \theta }\\
&\tan \theta =\dfrac {a}{x}\rightarrow x=\dfrac {a}{\tan \theta }
\end{flalign}

\begin{flalign}
&x=\dfrac {\cos \theta }{\sin \theta }a
\end{flalign}

\begin{eqnarray}
\dfrac {dx}{d\theta }&=&\dfrac {-\sin ^{2}\theta -\cos ^{2}\theta }{\sin ^{2}\theta }a\\
&=&-\dfrac {a}{\sin ^{2}\theta }
\end{eqnarray}

\begin{eqnarray}
H&=&\sum \Delta H\\
&=&\sum \dfrac {I\Delta x\sin \theta }{4\pi r^{2}}=\sum \dfrac {I \sin \theta}{4\pi r^{2}}\Delta x\\
&=&\int ^{\infty }_{-\infty}\dfrac {I\sin \theta }{4\pi r^{2}}dx\\
&=& \int ^{0}_{\pi }\dfrac { I \sin \theta }{4\pi \dfrac {a^{2}}{\sin ^{2}\theta }}\left( -\dfrac {a}{\sin ^{2}\theta }\right) d\theta\\
&=&\int ^{0}_{\pi }\dfrac {-I\sin \theta }{4\pi a}d\theta \\
&=&\dfrac {1}{4\pi a}\int ^{0}_{\pi }\left( -\sin \theta \right) d\theta \\
&=&\dfrac {1}{4\pi a}\left[ \cos \theta \right] ^{0}_{\pi }\\
&=&\dfrac {1}{4\pi a}\left( 1-\left( -1\right) \right) \\
&=&\dfrac {I}{2\pi a}\left[ A/m\right]
\end{eqnarray}

\begin{eqnarray}
H&=&\int ^{x_{2}}_{x_{1}}\dfrac {I\sin \theta }{4\pi r^{2}}dx\\
&=&\int ^{\alpha _{1}}_{\pi -\alpha _{2}}\dfrac {-I\sin \theta }{4\pi a}d\theta \\
&=&\dfrac {I}{4\pi a}\left[ \cos \theta \right] ^{\alpha _{1}}_{\pi -\alpha _{2}}\\
&=&\dfrac {I}{4\pi a}\left( \cos \alpha _{1}-\cos \left( \pi -\alpha _{2}\right) \right) \\
&=&\dfrac {I}{4\pi a}\left( \cos \alpha _{1}+\cos \alpha _{2}\right)\left[ A/m\right]
\end{eqnarray}

\begin{eqnarray}
H&=&4\times \dfrac {I}{4\pi \times \dfrac {l}{2}}\left( \cos \dfrac {\pi }{4}\times 2\right)\\
&=&2\dfrac {I}{\pi l}\left( \dfrac {2}{\sqrt {2}}\right) \\
&=&\dfrac {2\sqrt {2}I}{\pi l}\left[ A/m\right]
\end{eqnarray}

\begin{flalign}
&\sum ^{n}_{i=1}H_{i}\Delta l_{i}=H\cdot 2\pi a
\end{flalign}

\begin{flalign}
&H=\dfrac {I}{2\pi a}\left[ A/m\right] \\
&H\cdot 2\pi a=I\\
&\therefore \sum ^{n}_{i=1}H_{i}\cdot \Delta l_{i}=I
\end{flalign}

\begin{flalign}
&\oint _{C_{0}}Hdl=I
\end{flalign}

\begin{flalign}
&\oint _{C_{0}}\overrightarrow {H}\cdot \overrightarrow {t}dl=I\\
&\oint _{C_{0}}H\cos \theta dl=I
\end{flalign}

\begin{flalign}
&\oint _{C_{0}}Hdl=I\\
&H\cdot 2\pi r=I\\
&H=\dfrac {I}{2\pi r}\left[ A/m\right]
\end{flalign}

\begin{flalign}
&\oint _{C_{0}}Hdl=I\\
&H\cdot 2\pi r=I\\
&H=\dfrac {I}{2\pi r}\left[ A/m\right]
\end{flalign}

\begin{flalign}
&\dfrac {I}{\pi a^{2}}\times \pi r^{2}=I\dfrac {r^{2}}{a^{2}}\left[ A\right] \\
&\oint _{C_{0}}Hdl=I\dfrac {r^{2}}{a^{2}}\\
&H=\dfrac {Ir^{2}}{2\pi ra^{2}}\\
&H=\dfrac {Ir}{2\pi a^{2}}\left[ A/m\right]
\end{flalign}

\begin{flalign}
&\oint _{C_{0}}Hdl=NI\\
&H\times 2\pi a=NI\\
&H=\dfrac {NI}{2\pi a}\left[ A/m\right]
\end{flalign}

\begin{flalign}
&B=\mu _{0}H=\dfrac {\mu _{0}NI}{2\pi a}\left[ T\right] \left[ Wb/m^2\right]
\end{flalign}

\begin{flalign}
&\phi =BS=\dfrac {\mu _{0}NIS}{2\pi a}\left[ Wb\right]
\end{flalign}

\begin{flalign}
&\oint _{C_{1}}Hdl=0\\
&-H_{1}\times a+H_{2}\times a=0\\
&H_{1}=H_{2}
\end{flalign}

\begin{flalign}
&\oint _{C_{3}}Hdl=0\\
&H_{3}=H_{4}\\
&H_{3}=H_{4}=0\left[ A/m\right]
\end{flalign}

\begin{flalign}
&\oint _{C_{2}}Hdl=\left( na\right) I\\
&-H_{5}\times a+H_{6}\times a=naI\\
&0\times a+H_{6}\times a=naI\\
&H_{6}=nI\left[ A/m\right]
\end{flalign}

\begin{flalign}
&H=n\cdot I\left[ A/m\right]
\end{flalign}

\begin{eqnarray}
\Delta \phi &=& B \times c \Delta x\\
&=& \mu _{0}\dfrac {I}{2\pi x}c\Delta x
\end{eqnarray}

\begin{eqnarray}
\phi &=&\sum \Delta \phi \\
&=&\sum \dfrac {\mu _{0}Ic}{2\pi x}\Delta x\\
&=&\int ^{b}_{a}\dfrac {\mu _{0}Ic}{2\pi x}dx\\
&=&\dfrac {\mu _{0}Ic}{2\pi }\left[ \log \left| x\right| \right] ^{b}_{a}\\
&=&\dfrac {\mu _{0}Ic}{2\pi }\log \dfrac {b}{a}\left[ Wb\right]
\end{eqnarray}

\begin{flalign}
&F\propto I\\
&F\propto l\\
&F\propto B\\
&F\propto IBl
\end{flalign}

\begin{flalign}
&F=IlB\left[ N\right]
\end{flalign}

\begin{flalign}
&&F=Il(B \cdot \sin \theta)\left[ N\right]
\end{flalign}

\begin{flalign}
&Il = qv
\end{flalign}

\begin{flalign}
&F=IlB
&F=qvB
\end{flalign}

\begin{flalign}
&\overrightarrow {F}=\left( \overrightarrow {I}\times \overrightarrow {B}\right) l\\
&\overrightarrow {F}=\left( \overrightarrow {v}\times \overrightarrow {B}\right) q
\end{flalign}

\begin{flalign}
&T=rF
\end{flalign}

\begin{eqnarray}
T&=&\dfrac {a}{2}\times F\sin \theta \times 2\\
&=&aF\sin \theta \\
&=&abIB\sin \theta \left[ Nm\right]
\end{eqnarray}

\begin{flalign}
&T=SIB\sin \theta \left[ Nm\right]
\end{flalign}

\begin{flalign}
&T=SIB\sin \theta \left[ Nm\right]
\end{flalign}

\begin{flalign}
&m\alpha =F\\
m\dfrac {v^{2}}{r}=lVB\\
m\dfrac {v}{r}=eB\\
r=\dfrac {mv}{eB}\left[ m\right]
\end{flalign}

\begin{flalign}
&\omega =\dfrac {V}{r}=\dfrac {eB}{m}\left[ \mathrm{rad}/s\right]
\end{flalign}

\begin{flalign}
&n=\dfrac {\omega }{2\pi }=\dfrac {eB}{2\pi m}\left[ s^{-1}\right]
\end{flalign}

\begin{flalign}
&F_{2}=I_{2}lB,\\
&B_{1}=\mu _{0}H_{1}=\mu _{0}\dfrac {I_{1}}{2\pi a}\\
&F_{1}=\dfrac {\mu _{0}I_{1}I_{2}l}{2\pi a}\left[ N\right] \\
&F_{1}=F_{2}
\end{flalign}

\begin{flalign}
&E=vB\\
&J=qnv\left[ A/m^{2}\right] \\
&v=\dfrac {J}{qn}\left[ m/s\right] \\
&\therefore \cdot E=\dfrac {J}{qn}B\left[ V/m\right]
\end{flalign}

\begin{flalign}
&\dfrac {E}{J}=\dfrac {1}{qn}B\\
\end{flalign}

$\dfrac {1}{qn}$をホール定数といい,$R_H$とおく.

\begin{flalign}
&\dfrac {E}{J}=R_{H}\times B\\
&\therefore B=\dfrac {1}{R_{H}}\cdot \dfrac {V_{H}/d}{I/dl}\left[ T\right]
\end{flalign}

\begin{flalign}
&F=IlB\left[ N\right]
\end{flalign}

\begin{flalign}
&W=-F_x=-IlBx\left[ J\right]
\end{flalign}

\begin{flalign}
&\phi = BS = B\left( xl \right)
\end{flalign}

\begin{flalign}
&W=-I\phi \left[ J\right]
\end{flalign}

\newpage

% 電磁誘導
\begin{flalign}
&e\propto \dfrac {\Delta \phi }{\Delta t}
\end{flalign}

\begin{flalign}
&e=-\dfrac {d\phi }{dt}\left[ N\right]
\end{flalign}

\begin{flalign}
&e=-n\dfrac {d\phi }{dt}\left[ N\right]
\end{flalign}

\begin{flalign}
&e=-\dfrac {d\psi }{dt}\left[ V\right]
\end{flalign}

\begin{flalign}
&\psi =n\phi \left[ Wb\right]
\end{flalign}

\begin{eqnarray}
e&=&-n\dfrac {d\phi }{dt}\\
&=&-n\dfrac {d}{dt}BS\\
&=&-n\dfrac {d}{dt}\left( B\left( y\cdot x\cdot \cos \theta \right) \right)
\end{eqnarray}

\begin{eqnarray}
e&=&-n\dfrac {d}{dt}\left( Byx\cos \omega t\right) \\
&=&-nBxy\left( -\sin \omega t\right) \cdot \omega \\
&=&n\omega xyB\sin \omega t \left[ V\right]
\end{eqnarray}

\begin{flalign}
&qE=\varepsilon vB\\
&E=v_{B}\left[ V/m\right]
\end{flalign}

\begin{flalign}
&e=vBl\left[ V\right]
\end{flalign}

\begin{flalign}
&\overrightarrow {e}=\left( \overrightarrow {v}\times \overrightarrow {B}\right) l\left[ V\right]
\end{flalign}

\begin{eqnarray}
W_{m}&=&F\times \left( v\times \Delta t\right) \\
&=&IlB\times v\times \Delta t\left[ J\right]
\end{eqnarray}

\begin{eqnarray}
W_{e}&=&e\times I\times \Delta t\\
&=&V\times B\times l\times I\times \Delta t\left[ J\right]
\end{eqnarray}

\begin{flalign}
&W_{m}=W_{e}
\end{flalign}




\newpage

% インダクタンス
\begin{flalign}
&H \propto I\\
&B= \mu_0 H \propto I\\
&\phi = BS \propto I
\end{flalign}

つまり$\phi \propto I$

\begin{flalign}
&\psi =n\phi \left[ Wb\right]
\end{flalign}

\begin{flalign}
&\psi \propto I
\end{flalign}

\begin{flalign}
&\psi = LI
\end{flalign}

\begin{flalign}
&n \phi = LI
\end{flalign}

\begin{eqnarray}
e&=&-n\dfrac {d\phi }{dt}=-\dfrac {d}{dt}\left( n\phi \right) \\
&=&-\dfrac {d\psi }{dt}\\
&=&-L\dfrac {dI}{dt}
\end{eqnarray}

\begin{flalign}
&
\end{flalign}

\begin{flalign}
&
\end{flalign}

\begin{flalign}
&
\end{flalign}

\begin{flalign}
&
\end{flalign}

\begin{flalign}
&
\end{flalign}

\begin{flalign}
&
\end{flalign}

\begin{flalign}
&
\end{flalign}

\newpage

% 磁性体
\input{./files/10/10.tex}
\newpage

% ベクトル解析
\input{./files/11/11.tex}
\newpage

% 電磁波
\input{./files/12/12.tex}
\newpage

\end{document}
